% Wikipedia : In mathematics, a Ford circle is a circle with centre at (p/q, 1/(2q^2)) and 
% radius 1/(2q^2), where p/q is an irreducible fraction, i.e. p and q are coprime
% integers.
% There is an interesting connection between Farey sequence and Ford circles.
% The following graphic is based on the Simple algorithm that you can find here
% http://en.wikipedia.org/wiki/Farey_sequence
% interesting information can be found here
% http://mathworld.wolfram.com/FordCircle.html

\documentclass{article}
\usepackage[dvipsnames]{xcolor}
\usepackage{tikz}
\newcommand{\drawcircle}[3]{%
   \pgfmathparse{1/(2*#2*#2)}
   \let\tempradius\pgfmathresult
   \draw[fill=#3] (#1/#2,\tempradius) circle (\tempradius);
}

\begin{document}

\def\fordN{5}
\def\fordA{0}
\def\fordB{1}
\def\fordC{1}
\let\fordD\fordN
\begin{center}\begin{tikzpicture}[scale=4]

\drawcircle{\fordA}{\fordB}{red}
\foreach \i/\c in
    {0/blue,1/yellow,2/green,3/magenta,
    4/orange,5/magenta,6/green,7/yellow,8/blue,9/red}{%
 \pgfmathparse{\fordN+\fordB}\global\let\fordK\pgfmathresult
 \pgfmathparse{floor(\fordK/\fordD)}\global\let\fordK\pgfmathresult
 \pgfmathparse{\fordK*\fordC}\global\let\fordE\pgfmathresult
 \pgfmathparse{\fordE-\fordA}\global\let\fordE\pgfmathresult
 \pgfmathparse{\fordK*\fordD}\global\let\fordF\pgfmathresult
 \pgfmathparse{\fordF-\fordB}\global\let\fordF\pgfmathresult
 \global\let\fordA\fordC
 \global\let\fordB\fordD
 \global\let\fordC\fordE
 \global\let\fordD\fordF
 \drawcircle{\fordA}{\fordB}{\c}
}

\end{tikzpicture}\end{center}
\end{document}
% name Ford Circle
% utf8
% pdflatex
% author Alain Matthes
