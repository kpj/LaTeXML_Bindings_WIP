\documentclass{article}

\usepackage{tikz}

\begin{document}

\noindent
\pgfmathadd{2}{6}\pgfmathresult \\
\pgfmathsubtract{2}{6}\pgfmathresult \\
\pgfmathneg{-3}\pgfmathresult \\
\pgfmathmultiply{2}{6}\pgfmathresult \\
\pgfmathdivide{2}{6}\pgfmathresult \\
\pgfmathreciprocal{3}\pgfmathresult \\
\pgfmathdiv{2}{6}\pgfmathresult \\
\pgfmathmod{-42}{5}\pgfmathresult \\
\pgfmathMod{-42}{5}\pgfmathresult \\
\pgfmathabs{-3}\pgfmathresult \\
\pgfmathe\pgfmathresult \\
\pgfmathln{5}\pgfmathresult \\
\csname pgfmathlog2\endcsname{5}\pgfmathresult \\
\csname pgfmathlog10\endcsname{5}\pgfmathresult \\
\pgfmathexp{5}\pgfmathresult \\
\pgfmathsqrt{5}\pgfmathresult \\
\pgfmathpow{5}{3}\pgfmathresult \\
\pgfmathfactorial{5}\pgfmathresult \\

\noindent
\pgfmathsin{0.2}\pgfmathresult \\

\begin{tikzpicture}
    \draw[red] (-1 + 2,0) -- (5 - 1,0);
\end{tikzpicture}

\end{document}
